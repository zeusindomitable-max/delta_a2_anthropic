\
\documentclass{article}
\usepackage{amsmath}
\usepackage{graphicx}
\title{Δa₂ Introspection Correlation: A Practical Metric for Early Representational Drift Detection}
\author{Zeus Indomitable}
\date{2025}
\begin{document}
\maketitle
\begin{abstract}
We introduce Δa₂, a scalar probe computed from activation covariance curvature, as a candidate early-warning signal for internal representational drift in large language models. We describe a reproducible evaluation pipeline and provide synthetic results demonstrating correlation with simulated detection.
\end{abstract}
\section{Introduction}
Motivation, related work, and the need for representational side-channels in alignment evaluation.
\section{Metric Definition}
Given activation matrix $h\in\mathbb{R}^{s\times d}$, compute covariance $C=\mathrm{Cov}(h)$ and define
\begin{equation}
a_2 = \frac{\operatorname{tr}(C^2)}{d},\quad \Delta a_2 = a_2^{post}-a_2^{pre}.
\end{equation}
\section{Experiments}
Describe synthetic injection experiments and model-backed protocol.
\section{Discussion and Limitations}
Calibration per architecture; interpret as signal, not verdict.
\section{Conclusion}
Δa₂ is a promising complement to output-level detectors and merits institutional evaluation.
\end{document}
